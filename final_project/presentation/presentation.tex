\documentclass{beamer}
\usepackage{xeCJK}
\usepackage[backend=biber]{biblatex}

\beamertemplatenavigationsymbolsempty
\usetheme{Warsaw}
\setCJKmainfont{ukai.ttc}
\setCJKsansfont{ukai.ttc}
\setCJKmonofont{ukai.ttc}
\addbibresource{reference.bib}
\setbeamertemplate{bibliography item}{\insertbiblabel}

\renewcommand{\bibfont}{\tiny}

\newcommand
{\footnotelink}
[1]
{\footnote{\texttt{\scriptsize #1}}}

\title{平行程式設計實務期末專題報告}
\author{陳風平}
\institute{國立台北科技大學 資訊工程系}
\date{110年06月18日}

\begin{document}

\begin{frame}
    \titlepage
\end{frame}

\begin{frame}
    \frametitle{目錄}
    \tableofcontents
\end{frame}

\section{研究目的}
\begin{frame}
    \frametitle{研究目的}
    \begin{itemize}
        \item 加速本人類神經網路 MATLAB 專案\footnotelink{https://github.com/phogbinh/handwritten-digit-recognition}(一下簡稱本專案)。
        \item 學習如何加速類神經網路訓練。
        \item 學習 MATLAB 平行與 GPU 函式庫。
    \end{itemize}
\end{frame}

\section{研究過程}
\begin{frame}
    \frametitle{第一階段:探索}
    \begin{itemize}
        \item 參考 Block Multiplication 加速方法\cite{blockmul}。
        \item 參考 \cite{imagenet} -- 用 C++/CUDA 在 GPU 做 convolution 的計算\footnotelink{https://code.google.com/archive/p/cuda-convnet}。% "We wrote a highly-optimized GPU implementation of 2D convolution and all the other operations inherent in training convolutional neural networks, which we make available publicly"
        \item 學習 MATLAB 平行與 GPU 計算教學系列\cite{partut}。
    \end{itemize}
\end{frame}

\begin{frame}
    \frametitle{第二階段:初步啟發}
    \begin{itemize}
        \item 整理原本程式碼 -- 類別、函式、命名等。
        \item Brainstorm 加速方法與問題(第 \ref{methods} 頁)。
        \item 決定專題策略:從實驗啟發。
     \end{itemize}
\end{frame}

\begin{frame}[label={methods}]
    \frametitle{第二階段:加速方法與問題}
    \begin{table}
        \centering
        \begin{tabular}{ p{0.3\textwidth} p{0.5\textwidth} }
            方法 & 問題 \\
            \hline
            把全部資料搬到 GPU 做計算 & 本專案最大矩陣為 $w^{2}_{47 \times 784}$,無法利用 GPU 矩陣相乘加速\cite{gpublog}\newline \\
            寫 C++ single precision 矩陣相乘 link 到 MATLAB 加速 & MATLAB 本身 BLAS 矩陣相乘已 highly-optimized\cite{matrixmulforum, matlabannounce},要花出很多功夫才能跟它速度相比\newline \\
            用 C++ 重寫本專案 & 要處理龐大資料儲存在記憶體裡面,要研究 C++ 線性代數圖書庫(如 Eigen3、GMTL 等\cite{stackoverflow})
        \end{tabular}
    \end{table}
\end{frame}

\begin{frame}
    \frametitle{第三階段:實驗}
    \begin{itemize}
        \item 使用 MATLAB Profiler 查看程式瓶頸\cite{profilervid}。
        \item 參考 MATLAB 提升效率建議\cite{matlabimprove}。
        \item 標註本專案可改做平行的部分\footnotelink{https://phogbinh.github.io/handwritten-digit-recognition/train\_fast.m}。
    \end{itemize}
\end{frame}

\begin{frame}
    \frametitle{第三階段:實驗結果}
    \begin{table}
        \centering
        \begin{tabular}{ p{0.55\textwidth} p{0.25\textwidth} }
            方法 & 加速時間(妙) \\
            \hline
            把 \texttt{layer} 屬性改變數,解開迴圈\newline & 250 \\
            把 \texttt{layer\_associates} 屬性改變數\newline & 150 \\
            在每個 mini-batch 用 \texttt{parfor} 平行\newline & 失敗\footnote{跑了 25 分鐘還沒訓練完成。} \\
            重用已配置記憶體的變數\cite{zeroforum}\newline & 1 \\
            取代全部全域變數\newline & 50 \\
            把全部資料搬到 GPU 做計算\newline & 失敗\footnote{跑了 40 分鐘還沒訓練完成。} \\
        \end{tabular}
    \end{table}
\end{frame}

\section{研究成果}
\begin{frame}
    \frametitle{研究成果}
    \begin{itemize}
        \item Demo 完整版\footnotelink{https://youtu.be/a7IcN0bq5Z8}。
        \item 成功把本專案加速了兩倍(原 911.8774 秒變 457.9892 秒)。 
        \item 觀察到 MATLAB 本身有 multi-threading(原 CPU 使用率 39\% 變 60\%)。
    \end{itemize}
\end{frame}

\begin{frame}
    \frametitle{參考文獻}
    \printbibliography[heading=none]
\end{frame}

\end{document}
