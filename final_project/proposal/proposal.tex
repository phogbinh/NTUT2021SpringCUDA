\documentclass[11pt, letterpaper, oneside]{article}
\usepackage[affil-it]{authblk}
\usepackage{xeCJK}

\setCJKmainfont{ukai.ttc}
\setCJKmonofont{ukai.ttc}

\title{\textbf{期末專題企劃書}}
\author{陳風平\thanks{學號:106590048}}
\affil{資訊工程系、國立台北科技大學、台北}
\date{\today}

\begin{document}

\maketitle

\section{Motivation}
近期深度學習非常夯,而大多數的模型都利用 GPU 來加速訓練。剛好我去年有寫出一份辨識手寫數字達 97\% 精準度的類神經網路模型\footnote{GitHub 專案:\texttt{https://github.com/phogbinh/handwritten-digit-recognition}}(單純使用 MATLAB 矩陣乘法),但那時候在 CPU 訓練要花很多時間,造成研究上的困擾。因此,我想針對這次專題用 CUDA 來加速該模型。

\section{Goal}
我真正的目的是想研究電腦科學家如何加速類神經網路。因此,我打算先自己想怎麼優化我的程式,再去參考別人的方法,整理出一份報告來跟大家分享。最後,我會選出一個實作又簡單,效果又不錯的方法來改進我的模型。這樣我不只學會別人的想法,也能接觸 MATLAB 平行的函式庫。

\section{Method}
因為 backpropagation 有前層依賴後層 gradient 的問題,我目前不清楚要怎麼做平行。不過不管是訓練還是預測,模型都會使用矩陣乘法,所以我會先往這個方向做研究,再來學習其他人的 paper。

\section{Expectation}
我希望加速後的訓練時間會比加速前快兩倍。

\end{document}
