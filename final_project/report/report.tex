\documentclass[12pt, letterpaper, oneside]{report}
\usepackage[affil-it]{authblk}
\usepackage{xeCJK}
\usepackage[backend=biber]{biblatex}

\setCJKmainfont{ukai.ttc}
\setCJKmonofont{ukai.ttc}
\addbibresource{reference.bib}

\renewcommand{\abstractname}{概要}
\renewcommand{\contentsname}{目錄}

\newcommand
{\footnotelink}
[1]
{\footnote{\texttt{\scriptsize #1}}}

\title{\textbf{平行程式設計實務期末專題報告}}
\author{陳風平\thanks{學號:106590048}}
\affil{資訊工程系、國立台北科技大學、台北}
\date{110年06月18日}

\begin{document}

\maketitle

\begin{abstract}
我們透過實驗加速兩倍一份 MNIST 前饋神經網路的 MATLAB 程式。
\end{abstract}

\tableofcontents

\chapter{研究動機}
近期的深度學習,不管是在學術界還是工業界,都是熱度頗高的技術之一。而大多數的深度學習模型都使用 NVidia 的 GPU,早就已經不拿 CPU 來做訓練了。

剛好我去年有自己寫出一份辨識手寫數字達 97\% 精準度的類神經網路模型\footnotelink{https://github.com/phogbinh/handwritten-digit-recognition}(一下簡稱本專案\footnote{本專案為單純使用矩陣乘法、無用任何其他外掛的 MATLAB 專案。})。但那時候每次在 CPU 執行都要花接近半個小時(訓練與測試),造成我研究上的困擾。因此,我想針對這次平行程式設計實務期末專題(一下簡稱本專題)應用 CUDA 來加速此模型。

\chapter{研究目的}
我真正的目的為研究電腦科學家如何加速類神經網路。

因此,我打算先自己想如何優化本專案,再去參考他人的方法。最後,我會選出一個實作又簡單、效果又不錯的方法來改進我的機器學習模型。這樣我不只學會他人的方法,也能接觸 MATLAB 平行與 GPU 函式庫,對我未來上碩士做學術研究有非常大的幫助。

\chapter{平行方法}
如本專案簡報的研究過程\footnotelink{https://phogbinh.github.io/NTUT2021SpringCUDA/final\_project/presentation/presentation.pdf}指出,此類神經網路無法做平行計算。我會在此章節帶過該簡報的重點。

\section{問題}
下方表格為本專案加速方法與問題:
\begin{table}[h!]
    \centering
    \begin{tabular}{ p{0.3\textwidth} p{0.5\textwidth} }
        方法 & 問題 \\
        \hline
        把全部資料搬到 GPU 做計算 & 本專案最大矩陣為 $w^{2}_{47 \times 784}$,無法利用 GPU 矩陣相乘加速\cite{gpublog}\newline \\
        寫 C++ single precision 矩陣相乘 link 到 MATLAB 加速 & MATLAB 本身 BLAS 矩陣相乘已 highly-optimized\cite{matrixmulforum, matlabannounce},要花出很多功夫才能跟它速度相比\newline \\
        用 C++ 重寫本專案 & 要處理龐大資料儲存在記憶體裡面,要研究 C++ 線性代數圖書庫(如 Eigen3、GMTL 等\cite{stackoverflow})
    \end{tabular}
    \caption{加速方法與問題}
\end{table}

\section{方法}
雖然本專案無法做平行計算,但透過實驗,我成功地把模型訓練的部分加速了兩倍:
\begin{table}[h!]
    \centering
    \begin{tabular}{ p{0.55\textwidth} p{0.25\textwidth} }
        方法 & 加速時間(妙) \\
        \hline
        把 \texttt{layer} 屬性改變數,解開迴圈\newline & 250 \\
        把 \texttt{layer\_associates} 屬性改變數\newline & 150 \\
        在每個 mini-batch 用 \texttt{parfor} 平行\newline & 失敗\footnote{跑了 25 分鐘還沒訓練完成。} \\
        重用已配置記憶體的變數\cite{zeroforum}\newline & 1 \\
        取代全部全域變數\newline & 50 \\
        把全部資料搬到 GPU 做計算\newline & 失敗\footnote{跑了 40 分鐘還沒訓練完成。} \\
    \end{tabular}
    \caption{實驗結果}
\end{table}

\chapter{研究成果}
我成功地把本專案模型訓練的部分加速了兩倍(原 911.8774 秒變 457.9892 秒\footnotelink{https://youtu.be/a7IcN0bq5Z8}) -- 達標 proposal 預期結果。

\chapter{結論心得}
透過本專題,我不只學會了 GPU 在類神經網路加速龐大矩陣乘法的用途,也上手了 MATLAB 平行與 GPU 函式庫。另外,我也順便學習到如何使用 \LaTeX 的 beamer 套件做簡報,非常有成就感。

\printbibliography

\end{document}
